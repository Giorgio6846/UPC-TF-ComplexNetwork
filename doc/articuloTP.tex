\documentclass[twocolumn]{article}

\usepackage[T1]{fontenc}
\usepackage[utf8]{inputenc}
\usepackage[spanish]{babel}
\usepackage{microtype}
\usepackage{lipsum,blindtext}
\usepackage{abstract}     
\usepackage{graphicx}
\usepackage{booktabs}
\usepackage{hyperref}
\hypersetup{colorlinks=true, urlcolor=blue, linkcolor=black, citecolor=black}

\usepackage[style=apa,backend=biber]{biblatex}
\DeclareLanguageMapping{spanish}{spanish-apa}
\addbibresource{refs.bib}

\newcommand{\keywords}[1]{\par\noindent\textbf{Palabras clave:} #1}

\title{Robustez y resilencia de la red comparativa}
\author{
    Mathias Hualtibamba, Giorgio Mancusi y Fabio Osorio \\
    Universidad Peruana de Ciencias Aplicadas \\
    Curso: Complex Networks \\
    2025-2 
}
\date{12 de octubre de 2025}

\begin{document}

\maketitle

\begin{abstract}
\blindtext[1]
\keywords{redes complejas, comparativas, analisis de resilencia}
\end{abstract}

\section{Introduccion}
\blindtext[2]

\section{Metodologia}
\blindtext[3]

\subsection{Datos}
El conjunto de datos utilizado está conformado por seis columnas: \textit{Año}, \textit{Persona}, \textit{Tipo de servicio}, \textit{Nombre de la tarea}, \textit{Modalidad} y \textit{Complejidad}. 
El dataset cuenta con un total de 10,384 registros, de los cuales aproximadamente 3,600 correspondían a duplicados. 
Estos registros duplicados fueron eliminados durante la etapa de limpieza, dado que no aportaban información adicional a la estructura del grafo y su presencia incrementaba innecesariamente el costo computacional en la construcción y análisis de la red.

\subsection{Construccion de la red}
\blindtext[1]

\subsection{Analisis y Metricas}
\blindtext[1]

\subsection{Herramientas}
Las herramientas utilizadas para la creacion del grafo fueron los siguentes:
\begin{itemize}
    \item Python 3.13
    \item NetworkX
    \item Pandas
    \item VSCode
\end{itemize}

\section{Resultados}
\blindtext[4]

\printbibliography

\section{Anexos}
El codigo, notebook y datos se encuentran en el siguente repositorio \parencite{repoCodigo}

\end{document}
