\documentclass[twocolumn]{article}

\usepackage[T1]{fontenc}
\usepackage[utf8]{inputenc}
\usepackage[spanish]{babel}
\usepackage{microtype}
\usepackage{lipsum,blindtext}
\usepackage{abstract}     
\usepackage{graphicx}
\usepackage{booktabs}
\usepackage{hyperref}
\hypersetup{colorlinks=true, urlcolor=blue, linkcolor=black, citecolor=black}

\usepackage[style=apa,backend=biber]{biblatex}
\DeclareLanguageMapping{spanish}{spanish-apa}
\addbibresource{refs.bib}

\newcommand{\keywords}[1]{\par\noindent\textbf{Palabras clave:} #1}

\title{Insertar Titulo}
\author{
    Mathias Hualtibamba, Giorgio Mancusi y Fabio Osorio \\
    Universidad Peruana de Ciencias Aplicadas \\
    Curso: Complex Networks \\
    2025-2 
}
\date{12 de octubre de 2025}

\begin{document}

\maketitle

\begin{abstract}
\blindtext[1]
\keywords{redes complejas, comunidades, centralidad, difusión, análisis estructural}
\end{abstract}

\section{Introduccion}
\blindtext[2]

\section{Metodologia}
\blindtext[3]

\subsection{Datos}
\blindtext[1]

\subsection{Construccion de la red}
\blindtext[1]

\subsection{Analisis y Metricas}
\blindtext[1]

\subsection{Herramientas}
\blindtext[1]

\section{Resultados}
\blindtext[4]

\section{Referencias}
\printbibliography

\section{Anexos}
\blindtext[1
]
\end{document}
